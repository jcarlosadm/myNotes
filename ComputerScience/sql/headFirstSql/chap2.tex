\chapter{O comando SELECT}
\label{chap:comando_select}

Usamos o comando \verb|SELECT| para encontrar informações relevantes em tabelas. Por exemplo:

\sqlcode{code/chap2/select01.sql}

O que o código acima faz é selecionar todas as colunas da tabela \verb|MEUS_CONTATOS|. Podemos melhorar isso.

\section{A cláusula WHERE}
\label{sec:clausula_where}

Com o comando \verb|SELECT| podemos usar a cláusula \verb|WHERE|, que serve para refinar uma seleção, informando ao sql alguma restrição. Por exemplo:

\sqlcode{code/chap2/select02.sql}