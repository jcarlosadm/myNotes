\chapter{Dados e tabelas}
\label{chap:dados_tabelas}

Uma \textbf{tabela} é um arranjo de \textbf{linhas} e \textbf{colunas}, onde as \textbf{colunas} são as categorias e as \textbf{linhas} são os dados de cada unidade (pessoa, objeto etc). Cada \textbf{coluna}, ou \textbf{campo}, possui um tipo. Cada \textbf{linha} representa um \textbf{registro}.

Um \textbf{banco de dados} é um local que guarda tabelas e outras estruturas relacionadas a essas tabelas. Todas as tabelas em um banco de dados precisam estar interligadas de alguma forma (precisam pertencer ao mesmo conceito).

Para criar uma base de dados, abra o console mysql e digite (o uso de maiúscula é uma boa prática, não uma obrigatoriedade):

\sqlcode{code/chap1/create01.sql}

E você pode mostrar a lista de base de dados com:

\sqlcode{code/chap1/show01.sql}

\begin{observacao}
Uma observação cabe aqui: você precisa ter o mysql instalado. Ok, você pode fazer isso com o pacote do \textit{XAMPP} ou instalar separadamente. Seja como for, execute o mysql com as opções \verb|-u root -p| e digite a senha do \textit{root} (apenas tecle \textit{ENTER} se não configurou).
\end{observacao}

Para começar a usar um banco de dados existente, é necessário usar o comando \verb|USE|:

\sqlcode{code/chap1/use01.sql}

\section{Criando uma tabela}
\label{sec:criando_tabela}

Para criar uma tabela em uma base de dados, usamos o comando \verb|CREATE TABLE|:

\sqlcode{code/chap1/create02.sql}

\verb|VARCHAR| é o tipo, neste caso \textit{string}. Para uma data usamos \verb|DATE|, para valores decimais usamos \verb|DEC|, para valores inteiros usamos \verb|INT|, e assim por diante. É preciso checar a sua versão do sql para ver quais nomes são utilizados.

Para descrever uma tabela em um banco de dados, use o comando \verb|DESC|:

\sqlcode{code/chap1/describe01.sql}

\section{Recriando tabelas}
\label{sec:recriando_tabelas}

O comando \verb|DROP| permite deletar algo, seja uma base de dados ou tabela. Por exemplo, para deletar uma base de dados faça:

\sqlcode{code/chap1/drop01.sql}

O seguinte comando deleta uma tabela:

\sqlcode{code/chap1/drop02.sql}

\begin{observacao}
Atenção: isso apagará o banco/tabela, assim como todos os dados! Tenha cuidado!
\end{observacao}

\begin{observacao}
O comando \verb|SOURCE| poderá executar um script sql com todos os comandos que ele tiver:

\bashcode{code/chap1/source01.sh}
\end{observacao}

Assim, depois de excluir uma tabela, você pode recriá-la usando o comando \verb|CREATE TABLE|. Mais adiante aprenderemos como adicionar campos sem precisar deletar toda a tabela.

\section{Inserindo valores em uma tabela}
\label{sec:inserindo_dados_tabela}

O comando \verb|INSERT| permite inserir valores em uma tabela. Vejamos um exemplo de uso:

\sqlcode{code/chap1/insert01.sql}

Se houvesse um campo numérico, o valor desse campo não poderia ter aspas.

Ao inserir valores, você pode omitir as colunas na declaração, mas os valores precisam estar na mesma ordem como no banco de dados. Ou então você pode alterar a ordem das colunas ou até omitir algumas, mas os valores precisam corresponder a ordem e as colunas que você declarou.

Para ver o conteúdo de uma tabela, use \verb|SELECT|:

\sqlcode{code/chap1/select01.sql}

Os campos que não foram preenchidos recebem \verb|NULL|. Para prevenir campos que devem ser preenchidos, use \verb|NOT NULL| no momento de criação da tabela.

\sqlcode{code/chap1/create03.sql}

Você pode definir um valor padrão para um campo com o comando \verb|DEFAULT|. Esse campo será preenchido com esse valor se não for definido um:

\sqlcode[firstline=1,lastline=7]{code/chap1/create04.sql}
\sqlcode[firstline=8,firstnumber=8]{code/chap1/create04.sql}






