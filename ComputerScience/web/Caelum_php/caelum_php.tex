\documentclass[12pt, a4paper]{article}

\usepackage[section]{minted}

\usepackage[brazilian]{babel}
\usepackage[T1]{fontenc}
\usepackage[utf8]{inputenc}

\usepackage[colorlinks=true]{hyperref}
\usepackage{color}

\title{Anotações da apostila Html, Css, Javascript e Php da Caelum (a partir do capítulo 8)}
\author{J Carlos Viana Filho}
\date{}

\definecolor{bg}{rgb}{0.95,0.95,0.95}

\newmintedfile[phpcode]{php}{
bgcolor=bg,
fontfamily=tt,
linenos=true,
numberblanklines=true,
numbersep=12pt,
numbersep=5pt,
gobble=0,
frame=leftline,
framerule=0.4pt,
framesep=2mm,
funcnamehighlighting=true,
tabsize=4,
obeytabs=false,
mathescape=false
samepage=false, %with this setting you can force the list to appear on the same page
showspaces=false,
showtabs =false,
texcl=false,
}

\newmintedfile[htmlcode]{html}{
bgcolor=bg,
fontfamily=tt,
linenos=true,
numberblanklines=true,
numbersep=12pt,
numbersep=5pt,
gobble=0,
frame=leftline,
framerule=0.4pt,
framesep=2mm,
funcnamehighlighting=true,
tabsize=4,
obeytabs=false,
mathescape=false
samepage=false, %with this setting you can force the list to appear on the same page
showspaces=false,
showtabs =false,
texcl=false,
}

\newmintedfile[bashcode]{bash}{
bgcolor=bg,
fontfamily=tt,
linenos=true,
numberblanklines=true,
numbersep=12pt,
numbersep=5pt,
gobble=0,
frame=leftline,
framerule=0.4pt,
framesep=2mm,
funcnamehighlighting=true,
tabsize=4,
obeytabs=false,
mathescape=false
samepage=false, %with this setting you can force the list to appear on the same page
showspaces=false,
showtabs =false,
texcl=false,
}

\begin{document}

% título

\makeatletter
\begin{center}
\textsl{\LARGE \@title}
\end{center}

\begin{flushright}
\emph{\@author}
\end{flushright}
\makeatother

% sumário

\tableofcontents

\newpage

% começo das anotações

\section{Rodar PHP no servidor local}

Para rodar o php no servidor local, entre na pasta e execute o comando:\\

\bashcode{bashcodes/initphp.sh}

O argumento \verb|-S| habilita o servidor php. Para testar, basta digitar \textit{localhost:8080} no navegador. Se passar \verb|0.0.0.0| no lugar do IP (que acima passamos \verb|localhost|), você estará habilitando todos os IPs da máquina. Isso quer dizer que o servidor estará acessível na rede local.

Um exemplo de código php:\\

\phpcode{phpcodes/basic.php}

\section{Modularização}

Por meio da função \verb|include()| é possível incluir código de outro arquivo php. Por exemplo, digamos que você tenha um arquivo de cabeçalho chamado \verb|cabecalho.php|:\\

\htmlcode{phpcodes/cabecalho.html}

Para inserir esse cabeçalho na página php, faça:\\

\phpcode{phpcodes/cabecalho.php}

\section{Variáveis}

As variáveis em php começam com \verb|$|. Você não declara o tipo de variável: esse tipo é definido de acordo com o conteúdo da variável. Você pode definir variáveis em php de forma a especificar arquivos \textit{css}, por exemplo. Neste contexto o php pode ser muito útil para definir que arquivos de estilo ou que códigos \textit{javascript} o navegador deverá carregar. Por exemplo:\\

\phpcode{phpcodes/var001.php}

\section{Formulários}

Para enviar objetos \verb|form|, usamos um de dois métodos: \verb|GET| e \verb|POST|. Enquanto \verb|GET| envia o formulário na \textit{url}, o método \verb|POST| envia o formulário no corpo da requisição. Por exemplo:\\

\htmlcode{phpcodes/form001.html}

Os dados enviados no formulários podem ser recebidos no php por meio das variáveis \verb|$_GET| e \verb|POST|, dependendo do método utilizado.

Para acessar o valor de um campo, precisamos do nome deste campo. No \textit{HTML}, definimos esse nome como segue abaixo:\\

\htmlcode{phpcodes/form002.html}

E acessamos esse campo com:\\

\phpcode{phpcodes/form003.php}

\section{Banco de dados e SQL}

Para usar banco de dados com php, em primeiro lugar instale o \textit{MySQL} ou baixe o \textit{phpMyAdmin}. Se escolher baixar o \textit{phpMyAdmin}, vá até a pasta e execute o php em uma porta diferente (como \verb|8000|.

\subsection{Conexão}

Para se conectar ao banco de dados, basta usar a função \verb|mysqli_connect|:\\

\phpcode{phpcodes/mysql001.php}

A função receberá onde se conectar (\textit{localhost}), o usuário(\textit{root}), a senha e o nome do banco de dados disponível(\textit{wd43}). Se tudo der certo, a variável \verb|$conexao| receberá a conexão aberta.

\subsection{A busca}

Com a conexão aberta, podemos realizar buscas no banco de dados. A função \verb|mysqli_query| permite isso. Ela recebe a conexão aberta e a SQL da busca:\\

\phpcode{phpcodes/mysql002.php}

A variável \verb|$dados| receberá seja lá o que a busca retornar.

\subsection{Dados no PHP}

Para usar esses dados no PHP, podemos utilizar a função \verb|mysqli_fetch_array|, que apenas passa os dados para a forma de um array:\\

\phpcode{phpcodes/mysql003.php}

\subsection{Refinando a busca}

Há muitas opções no SQL para se refinar buscas. A cláusula \verb|WHERE| é uma delas, que faz com que só retornem dados que atendam uma certa especificação.\\

\phpcode{phpcodes/mysql004.php}




\end{document}
























