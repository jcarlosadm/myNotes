%pdflatex -shell-escape headFirst_C.tex

\documentclass[12pt, a4paper]{article}
\usepackage{minted}
\usepackage[brazilian]{babel}
\usepackage[T1]{fontenc}
\usepackage[utf8]{inputenc}
\usepackage[colorlinks=true]{hyperref}
\usepackage{color}

\title{Anotações do livro Head First C}
\author{J Carlos Viana Filho}
\date{14 de agosto de 2014}

\definecolor{bg}{rgb}{0.95,0.95,0.95}

\newminted{c}{bgcolor=bg,linenos=true}

\begin{document}


% título

\begin{center}
\textsl{\LARGE Anotações do livro Head First C}
\end{center}

\begin{flushright}
\emph{J Carlos Viana Filho}
\end{flushright}

% sumário

\tableofcontents

\newpage

% começo das anotações

\section{Getting started with C}
\begin{flushright}
\textit{Capítulo 1 do livro, página 1}
\end{flushright}

\subsection{Saída}

Em \textbf{C} podemos usar uma saída formatada ou não formatada. Para saída formatada, usamos \verb|printf()|, e para saída não formatada usamos \verb|puts()|.

Há diversos caracteres especiais no uso de \verb|printf()| que nos permite formatar a saída como desejado. Veja o site \url{http://www.cplusplus.com} para mais detalhes.

\subsection{Executando o seu programa depois de compilado}

Você pode executar o seu programa logo após a compilação, usando o caractere especial \verb|&&|. Por exemplo:

\begin{verbatim}
gcc zork.c -o zork && ./zork
\end{verbatim}

Procure usar a flag \verb|-Wall| para compilar com avisos.

\subsection{Funções void e retornos vazios}

Você pode, numa função com retorno \verb|void| usar \verb|return|, apenas para controlar o fluxo da função. Por exemplo:\\

\begin{ccode}
void printDivision(float a, float b){
    if(b!=0){
        printf("%f.2 \n",(float)a/b);
        // apenas para finalizar a funcao
        return;
    }
    else{
        printf("denominador zero!\n");
    }
}
\end{ccode}

\section{Memory and pointers}
\begin{flushright}
\textit{Capítulo 2 do livro, página 41}
\end{flushright}

Como \textbf{C} passa argumentos por valor para funções, o uso de ponteiros é necessário para que os argumentos possam ser alterados no destino.

\subsection{Aritmética de ponteiros}

Você pode usar uma aritmética de ponteiros para percorrer um \textit{array}:\\

\begin{ccode}
int drinks[]={4,2,3};
printf("3rd order: %i drinks\n",drinks[2] );
// esse printf produz o mesmo resultado do printf acima
printf("3rd order: %i drinks\n", *(drinks + 2));
\end{ccode}

\subsection{Tenha cuidado com scanf}

Ao usar \verb|scanf()| para receber caracteres, procure limitar a quantidade de caracteres que irá receber. Por exemplo:\\

\begin{ccode}
char name[40];
scan("%39s",name);
\end{ccode}

Informamos que esperamos até 39 caracteres mais "\verb|\0|". Sem essa informação, a quantidade de caracteres informada pelo usuário pode ultrapassar o limite do \textit{array}, invandindo memória alheia, algo que chamamos de \textit{buffer overflow}, o que pode gerar problemas de segurança.

Ao invés de \verb|scanf()|, você pode usar \verb|fgets()|. Essa função recebe o ponteiro para o \textit{buffer}, que pode ser uma variável do tipo \verb|char|, depois o tamanho máximo da \textit{string}, e por último o local de onde estão vindo os dados (em geral usamos \verb|stdin| para o terceiro argumento).\\

\begin{ccode}
fgets(food, sizeof(food),stdin);
\end{ccode}

No código acima, \verb|fgets()| usa \verb|sizeof()| para obter o tamanho do \verb|array|. Porém, se usássemos um ponteiro no lugar de um \verb|array|, a função \verb|sizeof()| retornaria o tamanho do ponteiro, e não do seu conteúdo!

Também é uma boa idéia usar \verb|fgets()| para capturar espaços digitados pelo usuário. No caso de \verb|scanf()|, ela recebe todos os caracteres digitados dentro do limite até o primeiro caractere de espaço, onde ela para de receber.

Antes de \verb|fgets()| tínhamos \verb|gets()|, uma função que não se usa mais devido a problemas de segurança, pois não é possível definir um limite de caracteres a receber.

\subsection{Strings literals}

Uma variável que aponta para uma \textit{string literal} não pode ser usada para mudar o conteúdo da \textit{string}.\\

\begin{ccode}
char *cards = "JQR";
\end{ccode}

Mas você pode alterar uma \textit{string literal} se esta for guardada em um \verb|array|:

\begin{ccode}
char cards[] = "JQR";
\end{ccode}

\subsection{String Theory}

Para gerar um \verb|array| de \textit{strings}, você precisa de um \verb|array| bidimensional:\\

\begin{ccode}
char tracks[][80] = {
    "I left my heart in Havard Med School",
    "Newark, Newark - a wonderfull town"
};

puts(tracks[1]);

puts(tracks[0][0]);
\end{ccode}

Para manipulação de \textit{Strings} existe uma biblioteca chamada \verb|string.h|, que permite comparação de strings, cópia de strings e outros tipos de manipulação.

\section{Advanced Functions}
\begin{flushright}
\textit{Capítulo 7 do livro, página 311}
\end{flushright}

Você pode criar ponteiros para funções:\\

\begin{ccode}
// Declara um ponteiro para funcao que retorna um int
// e recebe um int
int (*warp_fn)(int);
// o ponteiro recebe o endereco da funcao go_to_warp_speed
warp_fn = go_to_warp_speed;
// usa o ponteiro como se usasse a funcao original
(warp_fn)(4);
\end{ccode}

Também pode criar array de ponteiros para função:\\

\begin{ccode}
void(*p[])(void)={do_,re_,mi_};
int i=0;
for(i=0;i<3;i++)
    (p[i])();
\end{ccode}

\section{Static and dynamic libraries}
\begin{flushright}
\textit{Capítulo 8 do livro, página 351}
\end{flushright}

No arquivo fonte do projeto, você pode indicar \textit{headers files} entre os sinais \verb|<| e \verb|>|. O compilador sempre procura esses arquivos na pasta padrão do sistema para \textit{header files} (que no linux costuma ser \verb|/usr/include| e \verb|/usr/local/include|, esse segundo local geralmente para \textit{headers} de terceiros).

A opção \verb|-I| diz ao compilador qual a pasta onde se encontram os \textit{headers files} padrão. Para indicar a pasta atual onde o arquivo fonte se encontra, use essa opção com um ponto: \verb|-I.|. Para indicar uma pasta na pasta raiz com o nome \textit{myHeaderFiles}, use: \verb|-I/myHeaderFiles|.

Ao invés de usar diversos \textit{object files} na compilação, você pode usar um \textit{archive file}. Um \textit{archive file} contém diversos \textit{object files}. Criar um é fácil, basta usar o comando \verb|ar|:

\begin{verbatim}
ar -rcs libmyarchivefile.a object1.o object2.o
\end{verbatim}

O comando acima cria um arquivo \verb|libmyarchivefile.a| a partir dos arquivos objetos \verb|object1.o| e \verb|object2.o|. A opção \verb|r| significa que o arquivo \verb|.a| será atualizado se ele já existir. A opção \verb|c| significa que o arquivo será criado sem qualquer feedback, e a opção \verb|s| significa que será criado um índice no início do arquivo.

É importante que o \textit{archive file} comece seu nome com \textit{lib}. A explicação é que todos os \textit{archive files} são \textit{static libraries}. Além disso, ao usar esse arquivo, você informará somente a parte do nome após o \textit{lib}; o compilador juntará \textit{lib} com o que você informar.

Você precisa colocar esse \textit{archive file} em algum lugar. Pode ser um lugar padrão, que varia dependendo do seu sistema operacional (no \textit{linux} costuma ser \verb|/usr/local/lib|), ou pode usar uma opção do compilador para indicar onde seu \textit{archive file} se encontra. Neste segundo caso, usamos a opção \verb|-L| para indicar esse diretório, da mesma forma que a opção \verb|-I|.

Usar um \textit{archive file} é simples. Basta usar a opção \verb|-l|, seguido do nome do arquivo (esse nome pode conter o caminho até o arquivo). Por exemplo:

\begin{verbatim}
gcc test.c -L/myLibs -lmyarchivefile -o test
\end{verbatim}

O código acima indica que os \textit{archive files} se encontram no diretório \textit{myLibs}. Além disso, a opção \verb|-l| diz ao compilador para procurar pelo arquivo \textit{libmyarchivefile.a}. Veja que omitimos as partes \textit{lib} e \textit{.a} do nome do arquivo.

\begin{quote}
\rule{\textwidth}{1pt}
Obs: Isso nos permite baixar (ou compilar) qualquer biblioteca e usar no diretório do projeto, sem a necessidade de instalar no diretório padrão do sistema. Isso vale para \textit{header files} e \textit{archive files}.
\rule{\textwidth}{1pt}
\end{quote}

Após compilar o programa, não há como esse programa ser alterado sem ser recompilado. Isso por que as bibliotecas usadas são estáticas. Ao alterar um programa, é necessário recompilá-lo.

\subsection{Dynamic Liking}

Ao invés de gerar um executável com bibliotecas estáticas, você pode gerar bibliotecas dinâmicas. Essas bibliotecas dinâmicas são \textit{linkadas} em tempo de execução quando forem necessárias, o que significa que podem ser compiladas em separado sem a necessidade de recompilar o programa principal.

Gerar uma biblioteca dinâmica é um pouco mais trabalhoso, mas não tão difícil. Siga os seguintes passos:

\begin{enumerate}

\item Em primeiro lugar, gere um \textit{object file} com a opção \verb|-fPIC|. Isso cria um código de posição independente, que será carregado no tempo de execução. Alguns sistemas operacionais emitirão um alerta informando que essa opção não é necessária.

\begin{verbatim}
gcc -fPIC -c myFile.c -o myFile.o
\end{verbatim}

\item Crie a biblioteca dinâmica com a opção \verb|-shared|. Lembre-se que a biblioteca deve começar com a palavra \textit{lib}. A terminação dessa biblioteca depende do sistema operacional que você está programando. No \textit{Windows} será \verb|.dll| (\textit{dynamic link libraries}) para \textit{MinGW} e \verb|.dll.a| para \textit{Cygwin}. \textit{Linux} ou \textit{Unix} usam \verb|.so| (\textit{shared object files}). No \textit{Mac} será \verb|.dylib| (\textit{dynamic libraries}). Exemplo no \textit{Linux}:

\begin{verbatim}
gcc -shared myFile.o -o libmyFile.so
\end{verbatim}

\item Compile o seu programa usando a biblioteca dinâmica. O código abaixo primeiro gera um \textit{object file} a partir do arquivo \verb|main.c|, depois gera o programa \verb|main| usando a biblioteca \textit{libmyFile.so} (informar uma biblioteca dinâmica é feito do mesmo modo que uma biblioteca estática). A opção \verb|L.| indica que a biblioteca se encontra no mesmo diretório do projeto.

\begin{verbatim}
gcc -c main.c -o main.o
gcc main.o -L. -lmyFile -o main
\end{verbatim}

\end{enumerate}

Se a biblioteca estiver em um local padrão para bibliotecas, o programa pode ser executado normalmente. Se não, é necessário informar onde se encontra a biblioteca antes. Isso é feito da seguinte forma no \textit{Linux}:

\begin{verbatim}
export LD_LIBRARY_PATH=$LD_LIBRARY_PATH:/libs
\end{verbatim}

Isso irá fazer com que o compilador procure a biblioteca na pasta \textit{libs} a partir da raiz do projeto. Após isso, o programa irá executar sem problemas.

Digamos que seja necessário alterar o arquivo \verb|myFile.c|. Sem problemas. Execute os passos 1 e 2 da compilação, ou seja, não compile \textit{main} de novo. Basta gerar o arquivo \textit{libmyFile.so} que \textit{main} receberá o conteúdo da biblioteca sem problema nenhum!

\section{Processes and system calls}
\begin{flushright}
\textit{Capítulo 9 do livro, página 397}
\end{flushright}

\textit{System calls} são apenas funções que vivem dentro do \textit{kernel} do sistema operacional. A maior parte do código da biblioteca padrão de \textbf{C} depende dos \textit{System calls}. Um exemplo é a função \verb|system()|. Ela simplesmente manda uma \textit{string} para a linha de comando.\\

\begin{ccode}
// imprime "hi!" no terminal
system("echo `hi!'");
\end{ccode}

Você pode usar a função \verb|system()| para iniciar outros programas em seu código.

A função \verb|system()| pega uma \textit{string} e manda para o sistema executar algum processo ou comando. Porém, se houver alguma ambiguidade nessa \textit{string}, o sistema não conseguirá interpretar a mensagem. Existe uma função mais direta: \verb|exec()|. O que essa função faz é substituir um processo por outro. Um processo é um programa sendo executado na memória. Todo processo possui um \textit{PID}, uma identificação. Você pode usar o comando \verb|ps -ef| para listar esses processos e seus respectivos \textit{PIDs}.

Existem diversas funções \verb|exec()|, cada qual com um nome um pouco diferente dos demais e com o seu próprio conjunto de parâmetros. Mas é possível agrupar essas funções em dois grupos básicos: as funções de lista e as funções de \textit{array}. Todas as funções \verb|exec()| se encontram em \textit{unistd.h}.

\subsection{As funções de lista: execl(), execlp(), execle()}

As funções de lista aceitam argumentos de linha de comando como uma lista de parâmetros, como isso:

\begin{itemize}
\item O programa: o caminho completo até o programa (no caso de \verb|execl()| ou \verb|execle()|) ou o nome do programa (no caso de \verb|execlp()|).
\item O argumentos de linha de comando: você precisa fornecer a lista de argumentos de linha de comando para o programa. O primeiro argumento deverá ser o nome do programa (isso quer dizer que o nome do programa aparecerá duas vezes).
\item NULL: isso indicará quando termina a lista de argumentos de linha de comando.
\item Variáveis de ambiente: se a função \verb|exec()| termina o seu nome com a letra \textbf{e} (como \verb|execle()|), então você poderá, se desejar, passar um \textit{array} de variáveis de ambiente.
\end{itemize}

Exemplos:\\

\begin{ccode}
execl("/home/flynn/clu","/home/flynn/clu","paranoids",
"contract",NULL);
\end{ccode}

\begin{ccode}
execlp("clu", "clu", "paranoids", "contract", NULL);
\end{ccode}

\begin{ccode}
execle("/home/flynn/clu","/home/flynn/clu", "paranoids",
 "contract", NULL, env_vars);
\end{ccode}

\subsection{As funções de array: execv(), execvp(), execve()}

Se você tiver um \textit{array} de strings com os argumentos, então será mais fácil de passar esses argumentos.\\

\begin{ccode}
execv("/home/flynn/clu",my_args);
\end{ccode}

\begin{ccode}
execvp("clu",my_args);
\end{ccode}

\subsubsection{Passando variáveis de ambiente}

Todo processo possui um conjunto de variáveis de ambiente. Esses são valores que você observa com os comandos \verb|set| ou \verb|env| no terminal. Eles normalmente nos passam informações úteis como a localização da pasta \textit{home} ou onde encontrar os comandos. Programas \textbf{C} podem ler variáveis de ambiente com a função \verb|getenv()| (da biblioteca \textit{stdlib}).

\subsection{Controlando erros de chamadas do sistema}

As funções \textit{system calls} podem gerar algum erro. Por meio da biblioteca \textit{errno}, é possível capturar esse erro e mostrar alguma mensagem amigável na tela. Esse erro ficará guardado na variável global \verb|errno|. A função \verb|strerro()| (da biblioteca \textit{string}) pega a variável de erro e retorna uma \textit{string} sobre o significado do erro.\\

\begin{ccode}
if(execle("./diner_info","./diner_info","4",NULL,my_env)==-1){
    puts(strerror(errno));
    return -1;
}
\end{ccode}

\subsection{O comando fork()}

O problema com \verb|exec()| é que o programa principal é finalizado em sua chamada. Existe uma função que ao invés de substituir o processo atual, cria uma cópia dele: \verb|fork()|. Essa função cria uma cópia do processo atual, que será \textit{filho} deste. Então basta usar \verb|exec()| nessa cópia. Usar \verb|fork()| é simples:\\

\begin{ccode}
pid_t pid = fork();
\end{ccode}

Os retornos possíveis de \verb|fork()| podem ser: -1 se falhou, 0 se estiver em um processo \textit{filho} ou positivo se estiver no processo \textit{pai}. O que você precisa fazer é chamar \verb|exec()| no processo \textit{filho}. O processo \textit{pai} continuará com outras tarefas.

No \textit{Windows}, não podemos usar o método \verb|fork()|, a menos que esteja usando o \textit{Cygwin}. Há uma função similar denominada \verb|CreateProcess()|, uma versão melhorada de \verb|system()|.

\section{Interprocess Communication}

\begin{flushright}
\textit{capítulo 10 do livro, página 429}
\end{flushright}

\subsection{Redirecionando entradas e saídas}

Quando você executa um programa na linha de comando, é possível redirecionar a saída padrão do programa com o operador \verb|>|.

\begin{verbatim}
python ./rssgossip.py 'Snooki' > stories.txt
\end{verbatim}

A saída padrão é um dos três fluxos de dados padrão (\textit{standard data streams}). Os fluxos de dados padrão são a entrada padrão (\textit{stdin}), a saída padrão (\textit{stdout}) e o erro padrão (\textit{stderr}). A saída padrão é a janela, mas você pode fazer com que a saída seja enviada para outro lugar, como um arquivo.

\subsection{Descriptor Table}

Uma \textit{descriptor table} é uma tabela que relaciona a entrada/saída com o local em que essa entrada/saída será redirecionada. Todo processo possui uma \textit{descriptor table}. Vejamos uma tabela padrão (essa não é a aparência real, mas sim uma representação aproximada):

\begin{table}[h]
\centering
\begin{tabular}{|l|l|}
\hline
%\rowcolor[HTML]{EFEFEF} 
\textcolor{red}{\#} & \textcolor{red}{Data Stream}         \\ \hline
0  & The Keyboard        \\ \hline
1  & The Screen          \\ \hline
2  & The Screen          \\ \hline
3  & Database Connection \\ \hline
\end{tabular}
\end{table}

O valor 0 é para a entrada padrão, 1 é para a saída padrão e 2 é para o erro padrão. As outras linhas são para outros fluxos de dados que possam existir (como a conexão de banco de dados acima).

Quando você redireciona a saída padrão, o que você estará fazendo é alterar a linha da tabela onde se encontra o número 1. Ao invés de \textit{The screen}, podemos ter \textit{file.txt}.

O operador \verb|>| redireciona a saída padrão, enquanto que \verb|2>| redireciona a saída de erro padrão (você pode pôr todos os erros capturados em um arquivo log). O valor 2 é a entrada da \textit{descriptor table}. O operador \verb|<| serve para redirecionar a entrada padrão. Você pode ainda redirecionar a saída de erro padrão para a saída padrão:

\begin{verbatim}
./myProg 2>&1
\end{verbatim}

\subsection{O comando fileno()}

Toda vez que você abre um arquivo, o sistema operacional registra o novo item na \textit{descriptor table}. Você provavelmente já sabe ou tem um ideia de como abrir um arquivo usando a função \verb|fopen()|. Mas e para capturar o índice correspondente deste arquivo na \textit{descriptor table}? Use a função \verb|fileno()|.\\

\begin{ccode}
// my_file is a FILE type
int descriptor = fileno(my_file);
\end{ccode}

Com a função \verb|dup2()|, você pode duplicar o fluxo de dados de uma linha da tabela para outra linha, mesmo que essa outra linha já esteja ocupada. Por exemplo, digamos que o índice 3 esteja registrado para o fluxo de banco de dados, e o índice 4 esteja registrado para um arquivo aberto. O seguinte código fará com que o arquivo aberto seja registrado nos índices 3 e 4:\\

\begin{ccode}
dup2(4,3);
\end{ccode}

Lembre-se que os índices de 0 a 2 são reservados aos fluxos padrão. E podemos alterá-los também.

\subsection{Redirecionamento e a função fork()}

Quando se usa \verb|fork()|, cada processo \textit{filho} é um processo separado. Então, quando um desses processos \textit{filho} tentar escrever em um arquivo que outro processo estiver tentando escrever, poderão acontecer coisas inesperadas. Então precisamos de algo para fazer outros processos esperarem enquanto um processo escreve em um arquivo específico.

Podemos fazer isso com a função \verb|waitpid()|. Essa função não irá retornar algo a menos que o processo \textit{filho} morra. Para usar essa função, você deve usar a seguinte biblioteca:\\

\begin{ccode}
#include <sys/wait.h>
\end{ccode}

O seu uso:\\

\begin{ccode}
int pid_status;

// pid = process ID
// &pid_status = information about the process
// 0 = options
if(waitpid(pid,&pid_status, 0) == -1){
	printf("Error waiting for child process\n");
	exit(1);
}
\end{ccode}

Para capturar o \textit{status} da saída do processo \textit{filho}, usamos a macro \textit{WEXITSTATUS}.\\

\begin{ccode}
if(WEXITSTATUS(pid_status))
	puts("Error status non-zero");
\end{ccode}

\subsection{A função pipe()}

O máximo que você pode fazer até agora é esperar um processo terminar para voltar a executar o processo principal. E se você puder fazer com que dois processos se comuniquem? Com \verb|pipe()|, isso é possível.

O que essa função faz é registrar dois \textit{descriptors} na \textit{descriptor table}. Um de entrada e outro de saída. Com o que você aprendeu até agora, é possível redirecionar a entrada ou saída do \textit{pipe} para um fluxo padrão do processo. Vejamos como criar um \textit{pipe}:\\

\begin{ccode}
// fd[0] reads from the pipe e fd[1] writes to the pipe
int fd[2];

// try create the pipe
if(pipe(fd) == -1)
	puts("can't create the pipe");
\end{ccode}

Agora pode ser importante fechar um dos fluxos do \textit{pipe}, dependendo do objetivo de comunicação. Por exemplo, se somente o processo \textit{filho} puder enviar mensagens para o \textit{pai}, então o processo \textit{filho} fecha o fluxo de leitura:\\

\begin{ccode}
close(fd[0]);
\end{ccode}

No caso do processo \textit{pai}, se fecharia o fluxo de escrita. Agora você precisa redirecionar o fluxo de leitura/saída do \textit{pipe} para um fluxo padrão. Por exemplo, o processo \textit{filho} pode redirecionar o fluxo de escrita do \textit{pipe} para a saída padrão do processo:\\

\begin{ccode}
dup2(fd[1],1);
\end{ccode}

No caso do processo \textit{pai}, o redirecionamento seria o fluxo de leitura do \textit{pipe} para o fluxo de entrada padrão. Neste exemplo descrito, tudas as saídas do processo \textit{filho} seriam escutados pelo processo \textit{pai}. Esse último precisa apenas de uma maneira de capturar aquilo que escutar (como um \textit{loop}) e fazer algo com isso.

Quando o processo \textit{filho} morre, o comando \verb|fgets()| recebe um final-de-arquivo (a função retorna 0).

Os \textit{pipes} trabalham em apenas uma direção. Mas você pode criar dois \textit{pipes}, e ter então uma comunicação de duas mãos entre dois processos.

Dois processos que precisam se comunicar mas não são parentes entre si podem usar um \textit{pipe} nomeado para isso. Existe uma função \verb|mkfifo()| para criar \textit{pipes} assim.

(p 451)

\end{document}