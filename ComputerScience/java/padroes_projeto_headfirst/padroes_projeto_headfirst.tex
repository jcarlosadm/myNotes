\documentclass[a4paper,11pt]{report}

\usepackage[brazilian]{babel}
\usepackage[T1]{fontenc}
\usepackage[utf8]{inputenc}

\usepackage[colorlinks=true]{hyperref}
\usepackage{color}

\usepackage{minted}

\definecolor{bg}{rgb}{0.95,0.95,0.95}

\newmintedfile[javacode]{java}{
bgcolor=bg,
fontfamily=tt,
linenos=true,
numberblanklines=true,
numbersep=12pt,
numbersep=5pt,
gobble=0,
frame=leftline,
framerule=0.4pt,
framesep=2mm,
funcnamehighlighting=true,
tabsize=4,
obeytabs=false,
mathescape=false
samepage=false, %with this setting you can force the list to appear on the same page
showspaces=false,
showtabs =false,
texcl=false,
}

\title{Notas do livro Padrões de Projeto}
\author{J Carlos Viana F}
\date{}

\begin{document}

\maketitle

\tableofcontents

\chapter{Bem vindo aos padrões de projeto}
\label{chap:bemvindo}

\section{Padrão deste capítulo}

\textbf{Padrão Strategy.} Define uma família de algoritmos, encapsula cada um deles e os torna intercambiáveis. A estratégia deixa o algoritmo variar independentemente dos clientes que o utilizam.

\section{Princípios}

\begin{itemize}
\item Identifique os aspectos de seu aplicativo que variam e separe-os do que permanece igual.
\item Programe para uma interface, não uma implementação.
\item Dê prioridade à composição
\end{itemize}

\section{Como funciona o padrão Strategy}

Em primeiro lugar, separe o que permanece igual daquilo que muda. Por exemplo, digamos que você possui uma superclasse \textit{Duck} que possui comportamentos \textit{Fly()}, \textit{Quack()}, \textit{Display()} e \textit{Swing()}. Digamos que os comportamentos \textit{Fly()} e \textit{Quack()} é que mudam para cada subclasse de \textit{Duck}.

Neste caso, o que fazemos é substituir esses métodos que se alteram por uma instância de uma interface que possui implementações distintas para cada comportamento de \textit{Fly()} e \textit{Quack()}. Vejamos em primeiro lugar a interface \textit{FlyBehavior}:

\javacode{code/chap1/interface_flybehavior.java}

Agora vejamos a interface \textit{QuackBehavior}:\\

\javacode{code/chap1/interface_quackbehavior.java}

Veja que temos, em cada caso, a interface e as implementações. Você pode definir quantas implementações desejar, e isso já nos fornece alguma flexibilidade. Vejamos como fica a classe \textit{Duck}:

\javacode{code/chap1/Duck.java}

Agora, para mostrar o motivo de fazer tudo isso, vejamos como fica a classe derivada de \textit{Duck}:\\

\javacode{code/chap1/MallardDuck.java}

Agora vamos observar como uma instância de \textit{MallardDuck} é usada:\\

\javacode[firstline=1,lastline=3]{code/chap1/TestMallardDuck.java}
\javacode[firstline=4,firstnumber=4]{code/chap1/TestMallardDuck.java}

Agora, digamos que alguém queira uma classe que representa um pato que possa usar um foguete como método de vôo. Quais as alterações no projeto? Apenas implemente \textit{flyBehavior} em uma nova classe \textit{FlyRocketPowered} e faça a nova classe de pato instancializar esse comportamento de vôo. Observe que esse comportamento de vôo novo pode ser adotado por outras classes de pato (ou seja, existe reuso).

Também tem outra vantagem nisso tudo: a possibilidade de alterar o comportamento de vôo e grasnar \textit{durante o tempo de execução}. Você pode fazer o \textit{MallardDuck} voar com foguetes durante a execução do programa. Basta criar um método em \textit{Duck} que altere a variável de instância \textit{flyBehavior} para uma instância de outra classe de comportamento de vôo. É por esse e outros motivos que a composição sempre deve ser preferível à herança.

\chapter{Mantendo os seus objetos atualizados - o padrão Observer}
\label{chap:observer}

(p 46)

\end{document}