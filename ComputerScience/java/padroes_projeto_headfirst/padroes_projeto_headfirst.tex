\documentclass[a4paper,11pt]{report}

\usepackage[brazilian]{babel}
\usepackage[T1]{fontenc}
\usepackage[utf8]{inputenc}

\usepackage[colorlinks=true]{hyperref}
\usepackage{color}

\usepackage{minted}

\definecolor{bg}{rgb}{0.95,0.95,0.95}

\newmintedfile[javacode]{java}{
bgcolor=bg,
fontfamily=tt,
linenos=true,
numberblanklines=true,
numbersep=12pt,
numbersep=5pt,
gobble=0,
frame=leftline,
framerule=0.4pt,
framesep=2mm,
funcnamehighlighting=true,
tabsize=4,
obeytabs=false,
mathescape=false
samepage=false, %with this setting you can force the list to appear on the same page
showspaces=false,
showtabs =false,
texcl=false,
}

\title{Notas do livro Padrões de Projeto}
\author{J Carlos Viana F}
\date{}

\begin{document}

\maketitle

\tableofcontents

\chapter{Bem vindo aos padrões de projeto}
\label{chap:bemvindo}

\section{Padrão deste capítulo}

\textbf{Padrão Strategy.} Define uma família de algoritmos, encapsula cada um deles e os torna intercambiáveis. A estratégia deixa o algoritmo variar independentemente dos clientes que o utilizam.

\section{Princípios}

\begin{itemize}
\item Identifique os aspectos de seu aplicativo que variam e separe-os do que permanece igual.
\item Programe para uma interface, não uma implementação.
\item Dê prioridade à composição
\end{itemize}

\section{Como funciona o padrão Strategy}

Em primeiro lugar, separe o que permanece igual daquilo que muda. Por exemplo, digamos que você possui uma superclasse \textit{Duck} que possui comportamentos \textit{Fly()}, \textit{Quack()}, \textit{Display()} e \textit{Swing()}. Digamos que os comportamentos \textit{Fly()} e \textit{Quack()} é que mudam para cada subclasse de \textit{Duck}.

Neste caso, o que fazemos é substituir esses métodos que se alteram por uma instância de uma interface que possui implementações distintas para cada comportamento de \textit{Fly()} e \textit{Quack()}. Vejamos em primeiro lugar a interface \textit{FlyBehavior}:

\javacode{code/chap1/interface_flybehavior.java}

Agora vejamos a interface \textit{QuackBehavior}:\\

\javacode{code/chap1/interface_quackbehavior.java}

Veja que temos, em cada caso, a interface e as implementações. Você pode definir quantas implementações desejar, e isso já nos fornece alguma flexibilidade. Vejamos como fica a classe \textit{Duck}:

\javacode{code/chap1/Duck.java}

Agora, para mostrar o motivo de fazer tudo isso, vejamos como fica a classe derivada de \textit{Duck}:\\

\javacode{code/chap1/MallardDuck.java}

Agora vamos observar como uma instância de \textit{MallardDuck} é usada:\\

\javacode[firstline=1,lastline=3]{code/chap1/TestMallardDuck.java}
\javacode[firstline=4,firstnumber=4]{code/chap1/TestMallardDuck.java}

Agora, digamos que alguém queira uma classe que representa um pato que possa usar um foguete como método de vôo. Quais as alterações no projeto? Apenas implemente \textit{flyBehavior} em uma nova classe \textit{FlyRocketPowered} e faça a nova classe de pato instancializar esse comportamento de vôo. Observe que esse comportamento de vôo novo pode ser adotado por outras classes de pato (ou seja, existe reuso).

Também tem outra vantagem nisso tudo: a possibilidade de alterar o comportamento de vôo e grasnar \textit{durante o tempo de execução}. Você pode fazer o \textit{MallardDuck} voar com foguetes durante a execução do programa. Basta criar um método em \textit{Duck} que altere a variável de instância \textit{flyBehavior} para uma instância de outra classe de comportamento de vôo. É por esse e outros motivos que a composição sempre deve ser preferível à herança.

\chapter{Mantendo os seus objetos atualizados - o padrão Observer}
\label{chap:observer}

\section{Padrão desse capítulo}

\textbf{Padrão Observer.} Esse padrão define uma dependência um-para-muitos entre objetos de modo que quando um objeto mude de estado todos os seus dependentes sejam avisados e atualizados automaticamente.

\section{Princípios}

\begin{itemize}
\item Procure usar designs levemente ligados entre objetos que interagem
\end{itemize}

\section{Como funciona o padrão Observer}

Neste padrão o que temos é uma classe observável que pode ter o seu estado alterado a qualquer momento, e um conjunto de classes observadoras que estão interessadas na mudança de estado dessa classe observável. Então, como faremos para os observadores serem notificados de alguma mudança da classe observável?

Podemos usar duas interfaces para isso: \textit{Subject} (a classe observável) e \textit{Observer} (a classe observadora).

A interface \textit{Subject} possui três métodos básicos: \textit{registerObserver()}, \textit{removeObserver()} e \textit{notifyObserver()}. As duas primeiras apenas gerenciam quem de fato são os observadores desta classe. A terceira informa aos observadores interessados que alguma mudança ocorreu.\\

\javacode{code/chap2/Subject.java}

A interface \textit{Observer} possui apenas um método: \textit{update()}, que faz algo com a informação obtida da classe observável. Quem chama esse método, na verdade, é a classe observável. Então você deve implementar \textit{update()} de modo a executar algum bloco de código quando a classe observável mudar.\\

\javacode{code/chap2/Observer.java}

Uma nota sobre o método \textit{update()}: veja que ele recebe um parâmetro do tipo \textit{Object}. Você pode implementar esse método para que receba outros dados, mas talvez seja mais útil receber uma referência para a classe observável (e fazer o cast apropriado) de modo a obter somente os dados necessários.

A classe que implementar \textit{Subject} terá uma lista de objetos do tipo \textit{Observer}. Quando for necessário, basta disparar o método \textit{notifyObservers()}, que chamará o método \textit{update()} de cada objeto da lista. O método \textit{update()} irá atualizar cada objeto do tipo \textit{Observer} apropriadamente.

\section{API do pacote java.util}

A linguagem java possui uma API para esse padrão por meio de uma interface \textit{Observer} e uma classe \textit{Observable} de java.util.

A classe \textit{Observable} possui quatro métodos (você não precisa implementá-los): \textit{addObserver}, \textit{deleteObserver}, \textit{notifyObserver()} e \textit{setChange()}. Em primeiro lugar, a classe observável deve extender \textit{Observable} de java.util. Quando o estado da classe observável mudar, você deve chamar \textit{setChange()}, para logo em seguida chamar um dos métodos \textit{notifyObservers()}: com ou sem argumentos. Se for com argumentos, é do tipo \textit{Object} (são dados adicionais além daqueles que normalmente são enviados).

Cada classe observadora deve implementar a interface \textit{Observer} de java.util. Essa interface tem apenas um método: \textit{update(Observable obs, Object arg)}. Implemente-o do mesmo modo que na seção anterior.

O problema com essa API é que \textit{Observable} é uma classe, não uma interface. Porém, ela pode ser apropriada para diversas situações.

\section{Uso do padrão Observer}

Diversas APIs e bibliotecas usam esse padrão o tempo todo. Por exemplo, bibliotecas gráficas precisam saber quando o usuário clica em um botão.

\chapter{Decorando objetos - o padrão Decorator}



\end{document}