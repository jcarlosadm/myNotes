%pdflatex -shell-escape somatorios.tex

\documentclass[a4paper,11pt,fleqn]{article}

%\usepackage{minted}
\usepackage[brazilian]{babel}
\usepackage[T1]{fontenc}
\usepackage[utf8]{inputenc}
\usepackage[colorlinks=true]{hyperref}
%\usepackage{color}

\begin{document}

% titulo

\begin{center}
\textsl{\LARGE Apêndice A : Somatórios}\\

\textsl{\large ALGORITMOS - Cormen et al - 2ª ed}
\end{center}

\begin{flushright}
\emph{Anotações - página 835}
\end{flushright}

\begin{flushright}
\emph{J Carlos Viana Filho}
\end{flushright}

\section{Formulas e propriedades}

Dada a sequência finita $a_{1}, a_{2}, ..., a_{n}$, onde $n$ é um inteiro não negativo, pode-se escrever essa sequência como:

\[ \sum\limits_{k=1}^{n}a_{k}. \]

Se $n=0$, então o valor do somatório é definido como $0$.

Dada uma sequência $a_{1}, a_{2}, ...,$ a soma infinita $a_{1} + a_{2} + ...$ pode ser escrita como:


\[ \sum\limits_{k=1}^{\infty}a_{k}, \]

que é interpretada com o significado

\[\lim_{n \to \infty} \sum\limits_{k=1}^{n}a_{k}. \]

Se o limite existir, a série \textbf{\textit{diverge}}; caso contrário, ela \textbf{\textit{converge}}.

\subsection{Linearidade}

Para qualquer número real c e quaisquer sequências finitas $a_{1}, a_{2}, ..., a_{n}$ e $b_{1}, b_{2}, ..., b_{n}$,

\[ \sum\limits_{k=1}^{n}(ca_{k} + b_{k}) = c\sum\limits_{k=1}^{n}a_{k} + \sum\limits_{k=1}^{n}b_{k}. \]

\subsection{Série aritmética}

O seguinte somatório é uma série e tem o seguinte valor:

\begin{equation}
\sum\limits_{k=1}^{n}k = 1 + 2 + \ldots + n = \frac{1}{2} n(n + 1) = \mathcal{O} (n^{2}).
\end{equation}

\subsection{Somas de quadrados e cubos}

Temos os seguintes somatórios de quadrados e cubos:

\begin{equation}
\sum\limits_{k=0}^{n}k^{2} = \frac{n(n+1)(2n+1)}{6},
\end{equation}

\begin{equation}
\sum\limits_{k=0}^{n}k^{3} = \frac{n^{2}(n+1)^{2}}{4}.
\end{equation}

\subsection{Série geométrica}

Para o real $x \neq 1$, o somatório

\[ \sum\limits_{k=0}^{n}x^{k} = 1 + x + x^{2} + \ldots + x^{n} \]

é uma \textbf{\textit{série geométrica}} ou \textbf{\textit{exponencial}} e tem o valor

\begin{equation}
\sum\limits_{k=0}^{n}x^{k} = \frac{x^{n+1} - 1}{x - 1}.
\end{equation}

Quando o somatório é infinito e $|x|<1$, temos a série geométrica infinitamente decrescente

\begin{equation}
\sum\limits_{k=0}^{\infty}x^{k} = \frac{1}{1-x}.
\end{equation}

\subsection{Série harmônica}



\end{document}